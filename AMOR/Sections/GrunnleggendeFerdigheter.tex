\section*{Grunnleggende Ferdigheter} \label{Sec: GrunnleggendeFerdigheter}


Grunnleggende ferdigheter er integrert i kompetansemålene der de bidrar til utvikling av og er en del av fagkompetansen. I Anvendt Matematikk og Robotikk forstås grunnleggende ferdigheter slik:\\\\
\emph{Å kunne uttrykke seg muntlig og skriftlig} i Anvendt Matematikk og Robotikk innebærer å formulere logiske resonnementer, forklare en tankegang og sette ord på oppdagelser, ideer og hypoteser, i tillegg til å kommunisere med leverandører, kollegaer og fagfolk fra andre fagområder. Det vil si å stille spørsmål, delta i samtaler og drøftinger av matematiske situasjoner og problemer og argumentere for egne løsningsforslag. Det vil også si å delta i vurderinger og planlegging tilknyttet sikkerhet og valg av faglige løsninger. Å uttrykke seg skriftlig innebærer å planlegge arbeidsoppdrag og å dokumentere og rapportere inn utførte arbeidsoppdrag og avvik, samt å formulere matematiske bevis ved bruk av korrekt matematisk notasjon og logisk gyldige slutninger. I tillegg betyr det å skrive matematiske symboluttrykk og sette opp eller tegne tabeller, diagrammer, grafer og geometriske figurer.\\\\
\emph{Å kunne lese} i Anvendt Matematikk og Robotikk innebærer å trekke relevant informasjon ut av en tekst og kunne forstå symbolsk representasjon av roboter. Det betyr å forstå matematiske symboluttrykk og logiske resonnementer knyttet til fagspesifikke tekster. Det vil også si å forstå og tolke gjeldende regelverk og direktiver, i tillegg til organisert visuell informasjon, som DH-tabeller, grafer og geometriske figurer.\\\\
\emph{Å kunne regne} i Anvendt Matematikk og Robotikk er den mest grunnleggende ferdigheten. Det innebærer fortrolighet med valg av operasjon og fortrolighet med de ulike regneoperasjonene. Å regne betyr å benytte nye operasjoner som å regne ut determinant eller inversen til en matrise. Det vil også si å bruke teori og lineær algebra til å og vurdere rimeligheten av svar.\\\\
\emph{Å kunne bruke digitale verktøy} i Anvendt Matematikk og Robotikk innebærer å programmere, konfigurere og feilsøke på ulike robot-installasjoner. Det innebærer å bruke digitale verktøy til omfattende beregninger, simuleringer, og visualiseringer, samt å utlede, bearbeide og presentere matematisk informasjon i elektronisk form. I tillegg vil det si å vurdere robotens hensiktsmessighet og begrensninger i simuleringer, i tillegg til å gjøre informasjonssøk ved feilretting på både simuleringer og installasjoner.
