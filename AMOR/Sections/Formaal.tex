\section*{Formål} \label{Sec: Formaal}

	I et høyteknologisk samfunn hvor roboter løser stadig flere oppgaver, er fagets hensikt å gjøre elevene kvalifisert til å delta i fremste rekke av denne teknologiutviklingen.
	Uavhengig av videre karriærevalg vil en grunnleggende forståelse av teknologiens oppbygning og virkemåter være en nødvendig del av fremtidens kompetanse.
	Faget skal derfor forberede elevene på jobber innenfor robotikkfagfeltet, samt videre robotikkrelaterte studier på universitetsnivå.
	Da faget blant annet består av et halvt år med prosjektbasert læring hvor elevene jobber i team på ett halvårig prosjekt, skal det også fremme entrepenørskap og selvstendig læring, og være en virkelighetsnær simulering av arbeidslivet.
	Faget i seg selv har som formål å gi elevene teorigrunnlaget for å forstå hvordan en robot er bygd opp, da spesielt med tanke på software.
	Matematisk modellering er sentralt for å forstå oppbygningen av softwaren, og faget vil derfor introdusere endel matematiske konsepter som vanligvis dekkes på universitetsnivå.
	Disse konseptene er essensielle for å opprettholde og utvikle et høyteknologisk samfunn, og fagets egenart vil derfor også bidra til forståelse av matematikkens betydning i samfunnet, og til utvikling av argumenterende, analyserende og utforskende ferdigheter.
	Fagets formål inkluderer også et dannelsesperspektiv, som betyr at opplæringen blant annet skal fremme elevenes selvstendighet, samarbeids-, og kommunikasjonsevne.

\iffalse

	Arbeid med faget skal gi en innføring i logisk og analytisk tankegang med vekt på matematisk argumentasjon og framstillingsform, samtidig som elevene gjennom anvendelse får trening i sentrale metoder.
	Forståelse for internkontroll, helse, miljø og sikkerhet, verdiskapning i samfunnet, serviceinnstilling og bedriftens organisering skal ivaretas.
	Tverrfaglige læringsoppdrag skal danne grunnlag for videre fordypning og spesialisering og forberede elevene på livslang læring på arbeidsområder der den teknologiske utviklingen stiller krav til omstilling, endring og ny kompetanse.
	Opplæringen skal gi grunnleggende kunnskap om arbeidsmiljø og fremme selvstendighet og samarbeid med andre i og utenfor egen bedrift og eget fagområde samt evne til å kommunisere med brukere og kollegaer.



% TODO: skrive mer





Robotikk handler om mekaniske inretninger som kan handle på egenhånd, i enkelte tilfeller basert på hva de oppfatter om omgivelsene sine.
Robotikk er fremtidens teknologi.
Robotikk er en tverrfaglig gren av ingeniørfag og vitenskap som inkluderer maskinteknikk, elektronikk, informasjonsteknologi, datavitenskap og kunst
Robotikk er en viktig komponent i mange moderne produksjonsmiljøer.
Roboter kan utføre komplekse oppgaver, og de kan handle basert på omgivelsene sine.
Roboter løser stadig flere oppgaver i samfunnet, både for husholdninger, medisinsk industri, teknologiske bedrifter og forskning.
Roboter er den neste store platformen for læring og kommunikasjon
Roboter kan brukes i mange situasjoner og til mange formål, men i dag er mange brukt i farlige miljøer (inkludert bombe deteksjon og deaktivering), produksjonsprosesser, eller hvor mennesker ikke kan overleve (f.eks. I rommet, under vann, i høy varme og opprydding og inneslutning av farlige materialer og stråling).


I et samfunn som blir stadig mer digitalisert, automatisert og robotisert, er fagets hensikt å gjøre elevene kvalifisert til å delta i fremste rekke av teknologiutviklingen.
Faget utforsker hvordan roboter kan sanse og handle, og hvordan man kan bygge systemer som løser oppgaver.
I faget skal elevene få erfaring med å teste og feilsøke på ulike nivåer, og å programmere for å kunne kontrollere roboter
Faget skal gi kunskap om hvordan roboter kan styres og overvåkes ved å kombinere målemetoder, datateknologi, matematiske beregninger og påvirkningsinnretninger
Faget kombinerer datateknologi, elektronikk og mekanikk for å skape moderne tekniske løsninger i form av ulike robotapplikasjoner.


I et høyteknologisk samfunn hvor roboter løser stadig flere oppgaver kreves det kreativitet, fantasi og problemløsningsferigheter, for å ... komplekse systemer ...
Utviklingen av dagens og fremtidens teknologirike samfunn utfordrer både måten vi lærer på, og hvilken kompetanse som blir viktig.
Utviklingen innen robotikk er en sentral drivkraft til fornying av næringsliv og samfunn, og muliggjør et miljøvennlig fremtidssamfunn med meningsfylte jobber.
Med robotikk kan vi skape et miljøvennlig fremtidssamfunn med meningsfylte jobber, og mer tid til livets kreative aspekter.
Bruken av digital teknologi og robotikk øker i alle samfunnsområder, innen alt fra samferdsel til helse, og stiller nye krav til ferdigheter.
Kravet til omstilling og produktivitet i alle deler av samfunnet gjør at robotikk, automatisering og digitalisering blir stadig viktigere for vår moderne levemåte.
Automatisering og robotikk er en forutsetning for det moderne samfunnet.
Automatisering og robotikk er en sentral drivkraft til fornying av næringsliv og samfunn.
Automatisering og robotikk har en helt sentral plass i dagens arbeidsliv som en drivkraft i nærings- og samfunnsutviklingen.
Vi lever i et teknologirikt samfunn hvor de fleste forholder seg til en rekke digitale enheter daglig.


Programmering er en viktig ferdighet i dagens samfunn, som inngår i de fleste fagområder, og spesielt i robotikk.
Programmering åpner for å utforske komplekse og realistiske modeller av robotene.
Det gir også utvidede muligheter til å behandle store datamengder for trening av roboter.
I dette inngår prosessen fra å identifisere problemer og utforme mulige løsninger, til å lage kode som kan forstås av roboten, systematisk feilsøke og forbedre denne koden, og dokumentere løsningen på en forståelig måte.
Det omfatter alle nivåer fra å forutse og analysere hva et program skal gjøre, til å kjenne igjen mønstre, eksperimentere og evaluere mulige løsninger, og samarbeide med andre.
Summen av disse ferdighetene kalles algoritmisk tankegang.
Elevene skal få muligheten til å utvikle sin kreativitet og lage roboter ved hjelp av programmering.
Gjennom å lage programmer oppøver elevene også ferdigheter i å vurdere eget og andres arbeid, gi konstruktive tilbakemeldinger og samarbeide med andre.

\fi
