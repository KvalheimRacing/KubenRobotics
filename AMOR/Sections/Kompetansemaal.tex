\section*{Kompetansemål} \label{Sec: Kompetansemaal}

\subsection*{Anvendt Matematikk}

    Mål for opplæringen er at eleven skal kunne

    \begin{itemize}
        \item gjøre rede for de forskjellige tallsystemene opp til og med quarternionene
        \item regne med komplekse tall og quarternioner
        \item representere komplekse tall på matriseform
        \item representere en sekvens med rotasjoner i rommet med hensyn på fikserte og ikke fikserte aksekors
        \item representere rotasjoner i planet med komplekse tall
        \item beskrive rotasjonsmatriser og representere rotasjoner i rommet i form av euler vinkler, roll-pitch-yaw, akse-vinkel
        \item beskrive rotasjoner i rommet med quarternioner og kunne regne fram og tilbake mellom quarternionene og rotasjonene
        \item representere quarternioner på matriseform
        \item beskrive, redegjøre for og regne med vektorer og lineærtransformasjoner i planet og rommet
        \item bruke enkle regneregler for matriser som transponering, invers av en 2x2 og 3x3 matrise, addisjon, subtraksjon, multiplikasjon
        \item regne ut determinanter for 2x2 og 3x3 matriser og forklare determinantens betydning for lineærtransformasjonen
        \item regne ut og visualisere homogene transformasjoner
        \item for en lineær transformasjon, kunne regne med, og beskrive, begrepene lineær uavhengighet, basis og rang
        \item regne ut egenverdier og egenvektorer for 2x2 og 3x3 matriser
    \end{itemize}


\subsection*{Matematisk Modellering av Roboter}

    Mål for opplæringen er at eleven skal kunne

    \begin{itemize}
        \item bruke matematiske modelleringsverktøy som Matlab for å digitalt kunne representere en vilkårlig robot konfigurasjon
        \item redgjøre grunndig for av hva foroverkinematikk og inverskinematikk går ut på
        \item vise en overordnet forståelse av hva Jacobi matrisen beskriver, og regjøre for forskjellige singulariteter
        \item bruke Denavit-Hartenberg konvensjonen til å lage en matematisk representasjon av en vilkårlig robot
        \item løse inverskinematikk-problemer algebraisk, geometrisk, og numerisk/ved hjelp av digitale verktøy
        \item beregne Jacobian for en vilkårlig robot, og bruke denne til å utlede robotens singulariteter
        \item beregne dynamikken for en simpel robot
    \end{itemize}


\subsection*{Robotikk}

    Mål for opplæringen er at eleven skal kunne

    \begin{itemize}
        \item vise et reflektert forhold til roboters etiske påvirkning i samfunnet, samt ens egen rolle i dette
        \item presentere en helhetlig oversikt over robotikkfaget som fagfelt, og kunne gjøre rede for de forskjellige fagområders vinkling på robotikk(en) og hva det er for dem
        \item knytte den matematiske teorien opp imot et praktisk funksjonelt perspektiv
        \item redegjøre for en robots mekaniske aspekter og struktur
        \item bruke meta operativsystemet ROS til å programmere og styre en robot. I det inngår å
            \begin{itemize}
                \item opprette ROS noder
                \item bruke ROS topics og ROS services for å analysere data og tilstand for en robot
                \item opprette Publisher/Subscriber forhold
                \item simulere og visualisere
                \item bruke kommandovinduer, teksteditorer og operere i et linux miljø
            \end{itemize}
        \item benytte dynamikken for en robot i et reguleringsteknisk perspektiv for å kontrollere roboten
        \item programmere roboter med produsent spesifike programmeringsverktøy
    \end{itemize}


\subsection*{Prosjektarbeid}

    Mål for opplæringen er at eleven skal kunne

    \begin{itemize}
        \item lage fremdriftsplaner, mål og skjemaer for arbeidsoppgaver og materialbehov etter arbeidsbeskrivelser
        \item arbeide prosjektbasert med større robot prosjekter over lang tid, hvor det skal holdes gjevnlig møtevirksomhet igjennom prosjektledelse
        \item vurdere hvilke regelverk og normer som gjelder for arbeidet som skal utføres og anvende dette
        \item utføre risikovurdering og vurdere tiltak for ivaretakelse av person– og maskinsikkerhet
        \item programmere en robot selvstendig, enten ved bruk av produsent spesifik programmvare, eller meta operativsystemet ROS
        \item planlegge, sluttkontrollere og dokumentere arbeidet i en prosjektrapport ved bruk av typesettingssystemet LaTeX
        \item regjøre for begrepet \emph{kumulativ kunnskapsøkning} i en virksomhetssammenheng, og bidra med tiltak for å sikre dette
        \item utføre arbeid på robot installasjoner fagmessig, nøyaktig og i overensstemmelse med krav til helse, miljø og sikkerhet og rutiner for kvalitetssikring og internkontroll
        \item gjøre faglige vurderinger av eget og andres arbeid
    \end{itemize}
