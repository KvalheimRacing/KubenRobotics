\section*{Kompetansemål} \label{Sec: Kompetansemaal}

\subsection*{Anvendt Matematikk}

    Mål for opplæringen er at eleven skal kunne

    \begin{itemize}
        \item gjøre rede for de forskjellige tallsystemene opp til og med quarternionene, og redegjøre for quarternionenes relevans i dagens teknologi
        \item regne med komplekse tall og quarternioner, og representere komplekse tall samt quarternioner på matriseform
        \item beskrive rotasjonsmatrisenes oppbyning samt representere en sekvens med rotasjoner i rommet med hensyn på fikserte og ikke fikserte aksekors
        \item representere rotasjoner i planet med komplekse tall og representere rotasjoner i rommet i form av euler vinkler, roll-pitch-yaw, akse-vinkel
        \item beskrive rotasjoner i rommet med quarternioner og kunne regne fram og tilbake mellom quarternionene og rotasjonene
        \item beskrive, redegjøre for og regne med vektorer og lineærtransformasjoner i planet og rommet
        \item for 2x2 og 3x3 matriser, bruke enkle regneregler for matriser som transponering, invers, addisjon, subtraksjon, multiplikasjon, samt regne ut egenverdier og egenvektorer
        \item regne ut determinanter for 2x2 og 3x3 matriser og forklare determinantens betydning for lineærtransformasjonen
        \item for en lineær transformasjon, kunne regne med, og beskrive, begrepene lineær uavhengighet, basis og rang, samt regne ut og visualisere homogene transformasjoner
    \end{itemize}


\subsection*{Matematisk Modellering av Roboter}

    Mål for opplæringen er at eleven skal kunne

    \begin{itemize}
        \item bruke matematiske modelleringsverktøy som Matlab for å digitalt kunne representere en vilkårlig robot konfigurasjon
        \item redgjøre grunndig for hva foroverkinematikk og inverskinematikk går ut på, samt bruke Denavit-Hartenberg konvensjonen til å beregne foroverkinematikken til en vilkårlig robot
        \item løse inverskinematikk-problemer algebraisk, geometrisk, og numerisk/ved hjelp av digitale verktøy
        \item vise en overordnet forståelse av hva Jacobi matrisen beskriver, og regjøre for forskjellige singulariteter og deres påvirkning på roboten, i tillegg til å beskrive mulige forebyggende tiltak
        \item beregne Jacobi matrisen for en vilkårlig robot, og bruke denne til å utlede robotens singulariteter samt robotens dynamikk
    \end{itemize}


\subsection*{Robotikk}

    Mål for opplæringen er at eleven skal kunne

    \begin{itemize}
        \item vise et reflektert forhold til roboters etiske påvirkning i samfunnet, ens egen og andres rolle i dette, og drøfte argumenter for ulike synspunkt vedrørende konsekvenser
        \item presentere en helhetlig oversikt over robotikkfaget som fagfelt, og kunne gjøre rede for de forskjellige fagområders vinkling på robotikken og hva det er for dem
        \item redegjøre for en robots mekaniske aspekter og struktur, samt beskrive fordeler, ulemper og bruksområder relatert til disse egenskapene
        \item beskrive, redgjøre for, og drøfte kjente resultater og utfordringer i forskningsområdene baneplanlegging, beslutningstakning og navigering, samt sanntids lokalisering og kartlegging
        \item knytte den matematiske teorien opp imot et praktisk funksjonelt perspektiv og demonstrere dette ved prosjektarbeid
        \item programmere roboter med produsent-spesifike programmeringsverktøy og programmeringsspråk på en effektiv og fagmessig måte
        \item bruke programmeringsspråket C++ og meta operativsystemet ROS til å programmere og styre en robot. I det inngår å
            \begin{itemize}
                \item oprette ROS-Pakker og Cakin arbeidsområde
                \item opprette ROS noder
                \item bruke ROS topics og ROS services for å analysere data og tilstand for en robot
                \item opprette Publisher/Subscriber forhold
                \item simulere og visualisere
                \item bruke kommandovinduer, teksteditorer og operere i et linux miljø
            \end{itemize}
        \item benytte dynamikken til en robot i et reguleringsteknisk perspektiv for å kontrollere roboten
        \item beskrive måleprinsipper for ulike sensorer brukt i mobile roboter, herunder radar, lidar, ultralyd, kamera, sonar, RFID, og andre relevante sensorer for robotens eksterne omgivelser, i tillegg til interne tilstandssensorer som temperaturssensorer, enkodere, resolvere, potentiometere, hastighetssensorer, og andre relevante sensorer som bidrar til robotens virkelighetsforståelse
    \end{itemize}

\newpage

\subsection*{Prosjektarbeid}

    Mål for opplæringen er at eleven skal kunne

    \begin{itemize}
        \item lage fremdriftsplaner, mål og skjemaer for arbeidsoppgaver og materialbehov etter arbeidsbeskrivelser
        \item arbeide prosjektbasert med større robot prosjekter over lang tid, hvor det holdes gjevnlig møtevirksomhet igjennom prosjektledelse
        \item vurdere hvilke regelverk og normer som gjelder for arbeidet som skal utføres og anvende dette
        \item utføre risikovurdering og vurdere tiltak for ivaretakelse av person– og maskinsikkerhet, i tillegg til å gjøre faglige vurderinger av eget og andres arbeid
        \item programmere en robot selvstendig, enten ved bruk av produsent spesifik programmvare, eller meta operativsystemet ROS
        \item planlegge, sluttkontrollere og dokumentere arbeidet i en prosjektrapport ved bruk av typesettingssystemet LaTeX
        \item regjøre for begrepet \emph{kumulativ kunnskapsøkning} i en virksomhetssammenheng, og bidra med tiltak for å sikre dette
        \item utføre arbeid på robot installasjoner fagmessig, nøyaktig og i overensstemmelse med krav til helse, miljø og sikkerhet og rutiner for kvalitetssikring og internkontroll
    \end{itemize}
