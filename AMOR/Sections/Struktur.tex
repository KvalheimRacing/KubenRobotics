\section*{Struktur} \label{Sec: Struktur}

	\subsection*{Gjennomføring}

	Faget er todelt rundt jul, med en teoretisk del på høsthalvåret, og en prosjektdel på vårhalvåret. I teoridelen er det ønskelig å kombinere denne med mest mulig praksis, og grad av praksis vil avhenge av avsatte ressurser.


	\subsection*{Koordinering med andre fag}

	Anvendt Matematikk og Robotikk baserer seg på et samarbeid med følgende fag:

	\begin{itemize}
	\item Engelsk ENG1001 og ENG1003
	\item Matematikk MAT1013, REA3022 og REA3024
	\item Fysikk REA3004 og REA3005
	\item Data- og elektronikksystemer ELE1001
	\item Automatiseringssystemer ELE1003 og AUT2001
	\end{itemize}

	Samarbeidet med disse fagene sammenfaller med de enkelte fagenes læreplan,
	og man vil kunne oppnå en synergieffekt mellom fagene motivert av FYR (Fellesfag, Yrkesretting og Relevans).\\


	\subsubsection*{Engelsk}

		\begin{itemize}

			\item Gi et overordnet inblikk i hvordan roboter kan modelleres matematisk.
				\begin{itemize}
					\item Symbolsk representasjon
					\item Konfigurasjonsrom, Tilstandsrom og Arbeidsrom
					\item Foroverkinematikk
					\item Inverskinematikk
					\item Hastighetskinematikk
					\item Dynamikk
				\end{itemize}
			\item Gi et overordnet inblikk i roboters mekaniske aspekter.
			\begin{itemize}
				\item Kraftkilder
				\item Kontroll og Styringsmetoder
				\item Applikasjonsområder, Geometrisk oppbyggning og typiske konfigurasjoner
					\begin{itemize}
						\item Albuemanipulator (RRR) og 6-aksede roboter
						\item Kartesisk manipulator (PPP)
						\item Sylindrisk manipulator (RPP)
						\item SCARA (RRP)
						\item Paralelle manipulatorer
					\end{itemize}
			\end{itemize}
		\end{itemize}


		Det er også mulighet for å utføre dette med studiespesialiserende versjoner av engelskfaget som (ENG1002 mm.).


	\subsubsection*{Matematikk}

		\begin{itemize}

			\item[1T]
			\item[R1]
			\item[R2]

		\end{itemize}


	\subsubsection*{Fysikk}

		\begin{itemize}

			\item[F1]
			\item[F2]

		\end{itemize}

	\subsubsection*{Data- og elektronikksystemer}

		\begin{itemize}

			\item[Vg1] Kunne utvikle egne programmer ved hjelp av programmeringspråket C++. I dette inngår å bruke og forstå grunnleggende elementer som variabler, strukter, arrays, løkker og funksjoner. Det omfatter også bruk av biblioteker, feilsøking, generalisering, gjenbruk av løsninger.

		\end{itemize}


	\subsubsection*{Automatiseringssystemer}

		\begin{itemize}

			\item[Vg1] noe

			\item[Vg2] Kunne utvikle egne programmer ved hjelp av programmeringspråket C++. I dette inngår å bruke og forstå grunnleggende elementer som variabler, strukter, arrays, løkker, funksjoner, tester og brukerinteraksjon i terminal. Det omfatter også bruk av biblioteker, feilsøking, generalisering, gjenbruk av løsninger, samt elementer spesifikt for C++ som pekere og minnehåndtering. Elevene skal også kunne vurdere og analysere egen og andres programkode.

		\end{itemize}
