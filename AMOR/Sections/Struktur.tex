\section*{Struktur} \label{Sec: Struktur}

	Faget er et tredjeårs fag hvor man igjennom Vg1 og Vg2 utnytter synergieffekten som oppstår ved å trekke paraleller og sammenhenger mellom AMOR og andre fag i studieløpet.\\
	For riktig gjennomføring av faget kreves det derfor to års forberedelser, hvor detaljene rundt koordineringen med involverte fag er spesifisert under.

	\subsection*{Gjennomføring}

	Faget er lagt opp til å bestå av en teoretisk del på høsthalvåret, deretter en prosjektdel på vårhalvåret. I teoridelen er det ønskelig med teori og praksis om hverandre slik at teorien visualiseres, og elevene får relevante praktiske erfaringer blant annet for prosjektdelen. Grad av praksis vil avhenge av avsatte ressurser.



	\subsection*{Koordinering med andre fag}

	Anvendt Matematikk og Robotikk baserer seg på et samarbeid med følgende fag:

	\begin{itemize}
	\item Engelsk ENG1001 og ENG1003, eller lignende fag.
	\item Matematikk MAT1013, REA3022 og REA3024
	\item Fysikk REA3004 og REA3005
	\item Data- og elektronikksystemer ELE1001
	\item Automatiseringssystemer ELE1003 og AUT2001
	\end{itemize}

	Samarbeidet med disse fagene sammenfaller med de enkelte fagenes læreplan,
	og man vil kunne oppnå en synergieffekt mellom fagene motivert av FYR - Fellesfag, Yrkesretting og Relevans.\\


	\subsubsection*{Engelsk}

	Koordineringen med engelskfaget går over to år og inneholder følgende aspekter

		\begin{itemize}

			\item[Vg1] \begin{itemize}

							\item Gi et overordnet inblikk i hvordan roboter kan modelleres matematisk.

								\begin{itemize}
									\item Symbolsk representasjon
									\item Konfigurasjonsrom, Tilstandsrom og Arbeidsrom
									\item Foroverkinematikk
									\item Inverskinematikk
									\item Hastighetskinematikk
									\item Dynamikk
								\end{itemize}

							\item Gi et overordnet inblikk av roboters mekaniske aspekter.

								\begin{itemize}
									\item Kraftkilder
									\item Kontroll og Styringsmetoder
									\item Applikasjonsområder, Geometrisk oppbyggning og typiske konfigurasjoner

										\begin{itemize}
											\item Albuemanipulator (RRR) og 6-aksede roboter
											\item Kartesisk manipulator (PPP)
											\item Sylindrisk manipulator (RPP)
											\item SCARA (RRP)
											\item Paralelle manipulatorer
										\end{itemize}

								\end{itemize}

						\end{itemize}

			\item[Vg2] \begin{itemize}

							\item Gi et overordnet inblikk av roboters mekaniske struktur

							\begin{itemize}
								\item Manipulatorer
								\item Mobile Roboter
							\end{itemize}

							\item Gi et overordnet innblikk i robotikkens verden

							\begin{itemize}
								\item Industrielle Roboter
								\item Avanserte Roboter
								\item Sosiale Roboter
								\item Medisinske Roboter
								\item Personlige Roboter
							\end{itemize}

						\end{itemize}

		\end{itemize}


	\subsubsection*{Matematikk}

	Koordineringen med matematikk går over tre år og omfatter følgende fag

		\begin{itemize}

			\item[1T] Kunne utlede og anvende cosinussetningen, for eksempel med Pythagoras læressetning. Kunne bruke trigonometri og geometri i planet til å løse inverskinematiske problemer.
			\item[R1] Det er viktig at eleven oppnår en god og intuitiv forståelse av enhetssirkelen. Man vil i AMOR bygge videre på forståelsen rundt tallsystemene, det er derfor viktig at elevene har god og grunndig forståelse av de 1-dimensjonale tallsystemene $\N$, $\Z$, $\Q$ og $\R$. Eleven skal kunne bruke kjente geometriske resultater til å løse inverskinematiske problemer geometrisk, i tillegg til å løse inverskinematiske ligninger algebraisk.
			\item[R2] Utføre beregninger med 3-dimensjonale vektorer, og regne ut kryssprodukt for disse. Løsing av inverskinematikk algebraisk og geometrisk. Analysere og beregne dynamiske ligninger for en robot.

		\end{itemize}


	\subsubsection*{Fysikk}

	Koordineringen med fysikk går over to år og omfatter følgende fag

		\begin{itemize}

			\item[F1] Utforske girforhold, utvekslinger og kraft nødvendig for diverse robotikkapplikasjoner. Bruke bevaringslover for mekanisk energi til å utlede en dynamisk modell av en robotmanipulator.
			\item[F2] Utforske fysikken rundt myke robotapplikasjoner. Kunne utlede dynamiske modeller av mobile roboter.

		\end{itemize}


	\subsubsection*{Data- og elektronikksystemer}

	Koordinering med Data- og elektronikkfaget foregår i første klasse og omfatter følgende

		\begin{itemize}

			\item[Vg1] Kunne utvikle egne programmer ved hjelp av programmeringspråket C++. I dette inngår å bruke og forstå grunnleggende elementer som variabler, strukter, arrays, løkker og funksjoner. Det omfatter også bruk av biblioteker, feilsøking, generalisering, gjenbruk av løsninger.

		\end{itemize}


	\subsubsection*{Automatiseringssystemer}


		Koordineringen med Automatiseringsfaget går over to år og omfatter følgende aspekter


		\begin{itemize}

			\item[Vg1] Kunne redegjøre for de mest sentrale elementene innenfor reguleringsteknikk, da spesielt PID controllere og de forskjellige komponentenes egenskaper. Kunne beskrive prinsippene ved forskjellig type aktuatorer brukt i roboter, og beskrive måleprinsipper for sensorer for måling av fysiske størrelser som lys, lyd, varme og kraft, i tillegg til sensorer som enkodere, resolvere, potentiometere og hastighetssensorer.
			\item[Vg2] Kunne utvikle egne programmer ved hjelp av programmeringspråket C++. I dette inngår å bruke og forstå grunnleggende elementer som variabler, strukter, arrays, løkker, funksjoner, tester og brukerinteraksjon i terminal. Det omfatter også bruk av biblioteker, feilsøking, generalisering, gjenbruk av løsninger, samt elementer spesifikt for C++ som pekere og minnehåndtering. Elevene skal også kunne vurdere og analysere egen og andres programkode.

		\end{itemize}


\iffalse

\begin{itemize}
	\item redegjøre for de mest sentrale elementene innenfor reguleringsteknikk, da spesielt PID controllere og de forskjellige komponentenes egenskaper
	\item beskrive prinsippene ved forskjellig type aktuatorer brukt i roboter
	\item beskrive måleprinsipper for sensorer for måling av fysiske størrelser som lys, lyd, varme, kraft
	\item beskrive måleprinsipper for interne sensorer som enkodere, resolvere, potentiometere, hastighetssensorer
\end{itemize}

\fi
