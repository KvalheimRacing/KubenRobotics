\section*{Struktur} \label{Sec: Struktur}

	Samarbeidet med disse fagene sammenfaller med de enkelte fagenes læreplan,
	og man vil kunne oppnå en synergieffekt mellom fagene motivert av FYR.

	Dette faget baserer seg på et samarbeid med følgende fag:

	\begin{itemize}
	\item Engelsk ENG1001 og ENG1003
	\item Matematikk REA3022 og REA3024
	\item Fysikk REA3004 og REA3005
	\item Data- og elektronikksystemer ELE1001
	\item Automatiseringssystemer ELE1003 og AUT2001
	\end{itemize}

	\subsection*{Engelsk}

		\begin{itemize}

			\item Gi et overordnet inblikk i hvordan roboter kan modelleres matematisk.
				\begin{itemize}
					\item Symbolsk representasjon
					\item Konfigurasjonsrom, Tilstandsrom og Arbeidsrom
					\item Foroverkinematikk
					\item Inverskinematikk
					\item Hastighetskinematikk
					\item Dynamikk
				\end{itemize}
			\item Gi et overordnet inblikk i roboters mekaniske aspekter.
			\begin{itemize}
				\item Kraftkilder
				\item Kontroll og Styringsmetoder
				\item Applikasjonsområder, Geometrisk oppbyggning og typiske konfigurasjoner
					\begin{itemize}
						\item Albuemanipulator (RRR) og 6-aksede roboter
						\item Kartesisk manipulator (PPP)
						\item Sylindrisk manipulator (RPP)
						\item SCARA (RRP)
						\item Paralelle manipulatorer
					\end{itemize}
			\end{itemize}
		\end{itemize}


		Det er også mulighet for å utføre dette med studiespesialiserende versjoner av engelskfaget som (ENG1002 mm.).

	\subsection*{Matematikk}
		Matematikk R1 og R2.

	\subsection*{Fysikk}
		Fysikk F1 og F2.

	\subsection*{Data- og elektronikksystemer}
		Noe

	\subsection*{Automatiseringssystemer}
		Noe
