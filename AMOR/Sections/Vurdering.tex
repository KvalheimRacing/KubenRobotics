\section*{Vurdering} \label{Sec: Vurdering}



        Vurderingen i faget består av underveisvurderinger som både er karaktergivende og ikke-karaktergivende. De ikke-karaktergivende underveisvurderingene kommer som en form av prosjekthjelp i siste halvår. Det skal være 5 karaktergivende underveisvurderinger første halvår, som tilsammen teller 20\% av den endelige karakteren i faget. To innenfor hovedområdet Anvendt Matematikk, én innen hovedområdet Matematisk Modellering av Roboter, og to innen hovedområdet Robotikk. I tillegg skal det avholdes en eksamen etter endt førstehalvår, som teller 30\%. Denne eksamen skal omhandle hovedområdene Anvendt Matematikk, Matematisk Modellering av Roboter, og Robotikk. Eksamenen skal ha en valgfri eksamensform, det vil si at elevene skal kunne velge å gjennomføre eksamen muntlig eller skriftlig. De resterende 50\% av vurderingen i faget kommer som følge av et prosjektarbeid i siste halvår. Den karaktergivende vurderingen skal kun fremkomme etter endt prosjekt, men det skal minimum forekomme månedlige ikke-karaktergivende underveisvurderinger i form av prosjekthjelp og tilbakemeldinger i siste halvår.\\\\
        Det skal forekomme to midveisevalueringer, en i midten av hvert halvår, samt to sluttevalueringer, en etter hvert endt halvår. I midtveisevalueringer skal elevene evaluere sin egen prestasjon, progresjon og kunnskap i faget. I sluttevalueringene skal elevene evaluere faget samt sin egen innsats og kompetanse.
