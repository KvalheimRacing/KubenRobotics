\vspace{-40mm}

\section*{Hovedområder} \label{Sec: Hovedomraader}

Hovedområdene i faget vil springe ut fra matematikken og omfatte følgende temaer


\subsection*{Anvendt Matematikk}

    Hovedområdet handler om anvendelse av sentrale resultater fra lineær algebra, i tillegg til grunnleggende forsåelse for 1-,2-, og 4-dimensjonale tallsystemer. Det dreier seg om regning med komplekse tall og quarternioner, og homogene transformasjoner. Grunnleggende teknikker innebærer enkel matriseregning, utregning av determinanter og inverser av matriser. Videre handler hovedområdet om forskjellige representasjoner av rotasjoner både i planet og rommet. Sentrale begreper er lineærtransformasjoner, rotasjoner og quarternioner.


\subsection*{Matematisk Modellering av Roboter}

    Hovedområdet handler om å skape representasjoner av roboter gjennom matematisk modellering, både ved hjelp av digitale verktøy og manuelle beregniner.
    Videre handler hovedområdet om å løse kompliserte ligningssystem på forskjellige måter, blant annet ved geometrisk modellering. I hovedområdet inngår også beregning av uoppnåelige posisjoner og robotens romslige bevegelse. Sentrale begreper er forover-, invers-, og hastighetskinematikk, dynamikk og singulariteter.


\subsection*{Robotikk}

    Hovedområdet handler om å skape en helhetlig oversikt over robotikken, blant annet ved refleksjon og etikk, og også ved å knytte teorien opp imot praksis, da spesielt i form av programmering. Det dreier seg om programmering med produsentspesifik programmvare, programmering av reguleringssystemer basert på en robots dynamiske modell, og programmering ved bruk av meta operativsystemet ROS. Videre handler hovedområdet om å redegjøre for hvordan en robot sanser og forstår verden, som et grunnlag for diskusjon rundt hvordan roboter kan tilpasse beslutningstakingsprosessen til omgivelsene. Sentrale begreper er virkelighetsforståelse, beslutningstaking og programmering.


\subsection*{Prosjektarbeid}

    Hovedområdet handler om å gi elevene rom til å ufolde sin kreative side, og samtidig lære om prosjektarbeid fra et administrativt ledelsesperspektiv.
    Det dreier seg om om å arbeide med større robot prosjekter over lang tid, fremme entrepenørskap, og bidra til kumulativ kunnskapsøkning.
    Videre handler hovedområdet om å forholde seg til gjeldende regelverk, normer, HMS krav, og rutiner for kvalitetssikring og internkontroll.
    Det omfatter å planlegge, risikovurdere, sluttkontrollere og dokumentere, ved bruk av typesettingssystemet LaTeX.
    Sentrale begreper er dokumentering, prosjektarbeid, og kumulativ kunnskapsøkning.
