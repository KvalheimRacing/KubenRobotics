\section*{Hovedområder} \label{Sec: Hovedomraader}

Hovedområdene i faget vil springe ut fra matematikken og omfatte følgende temaer


\subsection*{Anvendt Matematikk}

    Hovedområdet handler om anvendelse av sentrale resultater fra lineær algebra, i tillegg til grunnleggende forsåelse for 1,2 og 4 - dimensjonale tallsystemer. Det dreier seg om regning med komplekse tall og quarternioner, og homogene transformasjoner. Grunnleggende teknikker innebærer enkel matriseregning, utregning av determinanter og inverser av matriser. Videre handler hovedområdet om forskjellige representasjoner av rotasjoner både i planet og rommet. Sentrale begreper er lineærtransformasjoner, rotasjoner og quarternioner.


\subsection*{Matematisk Modellering av Roboter}

    Hovedområdet handler om
    Det dreier seg om
    Videre handler hovedområdet om
    Det omfatter
    Sentrale begreper
    Grunnleggende teknikker
    I dette hovedområdet inngår

    Elevene skal kunne bruke matematiske modelleringsverktøy som Matlab for å digitalt kunne representere en vilkårlig robot konfigurasjon. Den matematiske modelleringen er en sentral del og bygger sammen med matematikken, grunnfundamentet i faget.  Sentrale begreper

    \begin{itemize}
        \item \textbf{Foroverkinematikk} - Elevene skal ha god forsåelse av hva foroverkinematikk går ut på. Elevene skal kunne bruke Denavit-Hartenberg konvensjonen til å lage en matematisk representasjon av en vilkårlig robot.
        \item \textbf{Inverskinematikk} - Elevene skal kunne løse inverskinematikk-problemer algebraisk, geometrisk, og numerisk/ved hjelp av digitale verktøy. Elevene skal også ha en god overordnet forståelse av hva inverskinematikkproblemet går ut på.
        \item \textbf{Hastighetskinematikk} - Elevene skal kunne beregne Jacobian for en vilkårlig robot, og bruke denne til å utlede robotens singulariteter. Elevene skal også ha en god overordnet forståelse av hva Jacobian beskriver, samt forskjellig type singulariteter.
        \item \textbf{Dynamikk} - Elevene skal kunne beregne dynamikken for en simpel robot, og knytte dette opp mot reguleringsteknikk.
    \end{itemize}

    Grunnleggende teknikker


\subsection*{Robotikk}

    Hovedområdet handler om anvendelse av sentrale resultater fra lineær algebra, .
    Det dreier seg om
    Videre handler hovedområdet om
    Det omfatter
    Sentrale begreper
    Grunnleggende teknikker
    I dette hovedområdet inngår

    Refleksjon og Etikk
    Programmering av roboter ved bruk av meta operativsystemet ROS.
     på en sånn måte at en robot som er satt opp med ROS skal kunne påvirke omverden. Det skal kunne opprettes ROS noder og elevene skal ha kjenskap til følgende begreper


\subsection*{Prosjektarbeid}

    Hovedområdet handler om anvendelse av sentrale resultater fra lineær algebra, .
    Det dreier seg om
    Videre handler hovedområdet om
    Det omfatter
    Sentrale begreper
    Grunnleggende teknikker
    I dette hovedområdet inngår

    I prosjektarbeidet har elevene mulighet til å enten lære produsentspesifik programmering, eller gå "makerveien" og jobbe videre med ROS. Elevene skal også lære å bruke LaTeX for prosjektrapproten. Elevene skal kunne arbeide med større robot prosjekter over lang tid, noe som fremmer entrepenørskap, samt bidrar til \emph{kumulativ kunnskapsøkning}.
     Sentrale begreper
