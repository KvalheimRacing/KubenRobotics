\section*{Hovedomraader} \label{Sec: Hovedomraader}


\subsection*{Lineær Algebra}

    \begin{itemize}
        \item Regneregler for matriser (transponert, ikke kommutativt, invers etc)
        \item Komplekse tall på matriseform
        \item Quarternioner på matriseform
        \item Determinanter
        \item Lineær Uavhengighet
        \item Basis og Rank
        \item Egenverdier og Egenvektorer
        \item mm.
    \end{itemize}


\subsection*{Tallsystemer}

    \begin{itemize}
        \item[2D:] Komplekse tall
        \item[4D:] Quarternioner
    \end{itemize}


\subsection*{Matematisk Modellering av Roboter}
    Elevene skal kunne løse problemer ved hjelp av programmering, også kjent som algoritmisk tankegang. Matematisk modellering av roboter er en sentral del av dette.


\subsection*{Refleksjon og Etikk}
    Elevene skal ha et reflektert forhold til roboters etiske påvirkning i samfunnet, samt deres egen rolle i dette.
    Elevene skal gjøre faglige vurderinger av eget og andres arbeid, da spesielt i form av prosjektrapporter.
    Elevene skal ha en helhetlig oversikt over robotikkfaget som fagfelt, og kunne gjøre rede for de forskjellige fagområdenes vinkling på robotikken.


\subsection*{ROS}
    Ros Stuff


\subsection*{Rotasjoner og homogene transformasjoner}
    Relevante elementer i Kap 2 i Spong. Her tar man også med Quarternioner


\subsection*{Foroverkinematikk}
    KK, DH konv ++


\subsection*{Inverskinematikk}
    Algebraisk, geometrisk, numerisk


\subsection*{Hastighetskinematikk}
    Jacobian, singulariteter etc


\subsection*{Dynamikk}
    Noen få små elementer her, avhenig av hvorvidt fysikk 2 er med i løpet.


\subsection*{Prosjektarbeid}
    Her har elevene mulighet til å lære produsentspesifik programmering, eller gå "makerveien" og jobbe videre med ROS. Elevene skal også lære å bruke LaTeX for prosjektrapproten. Elevene skal kunne arbeide med større prosjekter over lang tid. Dette fremmer entrepenørskap, samt bidrar til \emph{kumulativ kunnskap}.
