\section*{Hovedområder} \label{Sec: Hovedomraader}

Hovedområdene i faget vil springe ut fra matematikken og omfatte følgende temaer


\subsection*{Anvendt Matematikk}

    Hovedområdet handler om anvendelse av sentrale resultater fra lineær algebra, i tillegg til grunnleggende forsåelse for 1,2 og 4 - dimensjonale tallsystemer. Det dreier seg om regning med komplekse tall og quarternioner, og homogene transformasjoner. Grunnleggende teknikker innebærer enkel matriseregning, utregning av determinanter og inverser av matriser. Videre handler hovedområdet om forskjellige representasjoner av rotasjoner både i planet og rommet. Sentrale begreper er lineærtransformasjoner, rotasjoner og quarternioner.


\subsection*{Matematisk Modellering av Roboter}

    Hovedområdet handler om å skape representasjoner av roboter gjennom matematisk modellering, både ved hjelp av digitale verktøy og manuelle beregniner.
    Videre handler hovedområdet om å løse kompliserte ligningssystem på forskjellige måter, blant annet ved geometrisk modellering. I hovedområdet inngår også beregning av uoppnåelige posisjoner og robotens romslige bevegelse. Sentrale begreper er foroverkinematikk, inverskinematikk, hastighetskinematikk, dynamikk og singulariteter.


\subsection*{Robotikk}

    Hovedområdet handler om
    Det dreier seg om
    Videre handler hovedområdet om
    Det omfatter
    Sentrale begreper
    Grunnleggende teknikker
    I dette hovedområdet inngår

    Refleksjon og Etikk
    Programmering av roboter ved bruk av meta operativsystemet ROS.
     på en sånn måte at en robot som er satt opp med ROS skal kunne påvirke omverden. Det skal kunne opprettes ROS noder og elevene skal ha kjenskap til følgende begreper


\subsection*{Prosjektarbeid}

    Hovedområdet handler om
    Det dreier seg om
    Videre handler hovedområdet om
    Det omfatter
    Sentrale begreper
    Grunnleggende teknikker
    I dette hovedområdet inngår

    I prosjektarbeidet har elevene mulighet til å enten lære produsentspesifik programmering, eller gå "makerveien" og jobbe videre med ROS. Elevene skal også lære å bruke LaTeX for prosjektrapproten. Elevene skal kunne arbeide med større robot prosjekter over lang tid, noe som fremmer entrepenørskap, samt bidrar til \emph{kumulativ kunnskapsøkning}.
     Sentrale begreper
