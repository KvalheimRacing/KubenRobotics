\section{Forberedelser} \label{Sec: Forb}


    \begin{itemize}

        \item Før året starter burde det ses på og vurderes:

            \begin{itemize}
                \item Bruk av eksterne foredragsholdere som et motivasjonstiltak, samt et tiltak for bidrag til et kunnskapsnettverk for elevene.
                \item Ekskursjoner til relevante robotikkmiljøer og eller fabrikker som bruker roboter. Dersom det fins ressurser nok burde også robotmesser i Asia og eller Silicon Valley vurderes. Dette for å bygge et kunnskapsnettverk for elevene og fagmiljøet på Kuben.
                \item Læringsressurser og strategi for faget, i henhold til vurderinger og feedback fra elevene, samt erfaringer gjordt av fjordårets faglærere.
                \item Bruk av Student Response Systems (SRS) som Mentimeter og Kahoot. En eventuell bruk burde forberedes godt og samspille med fagets læreplanmål.
                \item Bruk av roboten AV1 som et tiltak ved sykefravær.
                \item Bruk av videoopptak/podcast som et tiltak for fravær, repetisjon og eventuelt inspirasjon for andre.
            \end{itemize}


        \item Det er planlagt å bruke en robot i en lysgitterkube per andre elev det første halvåret. Lysgitterkubene med robotene inni er anslått til å bli 80x80x80cm, og de skal være fastmontert på en pult, på en sånn måte at to elever kan dele samme pult. Disse pultene må være ferdig satt opp før året starter.

        \item Alle praktiske oppgaver som skal gjøres første halvår som er ment å visualisere teorien, må forberedes og planlegges før årsstart. Disse praktiske oppgavene utføres på robotene som står på elevens pulter, og oppgavene må derfor testes før årsstart.

        \item Dersom det er mulig burde det legges opp til å lage oppgaver i denne teoridelen slik at man oppnår \emph{kumulativ kunnskap}. Elevene må dermed sette seg inn i eksisterende kodebase, i tillegg til at de jobber videre på den med oppgavene de gjør. Dette vil ikke være mulig å gjøre fullt ut hvert år (på de 8 små robotene), da de samme teorikonseptene skal dekkes hvert år, men det burde vurderes om det er mulig å gjøre i noen grad.

        \item En prosjektmal i LaTeX skal ferdigstilles før året starter, slik at elevene lett kan bygge videre på LaTeX koden i sin egen rapport. Prosjektmalen skal brukes på vårhalvåret.

        \item Oppgavebeskrivelsen for de 5 prosjektene som skal jobbes med siste halvår skal være ferdig laget før året starter. Det er også en fordel å forberede fler enn 5 prosjekter, slik at elevene kan velge prosjekter de er motiverte for.

        \item Relevante robotikkmiljøer, fabrikker og messer er som følger:

            \begin{itemize}
                \item Intervensjonssenteret Rikshospitalet
                \item Halodi
                \item FFI (Forsvarets Forskningsinstitutt)
                \item ROBIN (Robotikk og Intelligente Systemer)
                \item Robotnor
                \item N-link
                \item Saga robotics
                \item Diverse fabrikker i industriparken på raufoss
                \item Sintef?
                \item iREX (International Robot Exhibition)
                \item WRC (World Robot Conference)
                \item ICRA (International Conference on Robotics and Automation)
                \item Robotex
                \item Techcrunch Sessions: Robotics
                \item mm.
            \end{itemize}

        \item Det skal forekomme minst ett møte mellom faglærere før året starter, for å planlegge kurset og ta høyde for momentene redgjort for i dette dokumentet og i læreplanen. Dokumentet skal også oppdateres årlig ihenhold til vurderinger, erfaringer og feedback. Dette for å legge til rette for en \emph{kumulativ prosess} i arbeidet med faget.

    \end{itemize}
