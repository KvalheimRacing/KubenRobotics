\section*{Engelskfaglig Koordinering} \label{Sec: Engelskfaglig_Koordinering}

AMOR vil basere seg på forkunnskaper gitt i engelskfagene på 1. og 2. trinn.
Disse forkunnskapene er introduserende elementer som gir et overblikk over robotikkfaget, og forbereder elevene på temaene som dekkes i AMOR.
Kuben vil primært forholde seg til «Spong et al. - Robot Modelling and Control» og «Siciliano et al. - Robotics Modelling, Planning and Control» i gjennomførelsen av AMOR, og disse bøkene vil kunne fungere som pensummateriale for engelskfaget.
Da Siciliano er noe mer avansert, ser man helst at relevante elementer i Spong dekkes på vg1, mens relevante elementer i Siciliano dekkes på vg2.
Deler av Spong som inneholder matematikk kan erstattes av vedlegg 2 og vil være tilgjengelig for engelsklærer til bruk som pensummateriale.
%% husk å definere, pluss lage vedlegg 2

Som et pedagogisk virkemiddel, og et motivasjonstiltak for studentene foreslås det å ha en muntlig høring med elevene, etter stoffet i engelskfaget er dekket.

	\subsection*{Muntlig høring}
			Her ser man for seg en 1 til 1 dialog mellom en elev og en faglært, hvor eleven blir hørt i stoffet, med engelsklærer tilstede.
			Eleven har mulighet til å stille spørsmål til teksten og samtalen burde bære preg av kontinuerlig dialog. Dette vil gi muntlige vurderingsmuligheter for engelsklærer. Faglærer forbereder kontrollspørsmål for dialogen i samsvar med engelsklærer.
