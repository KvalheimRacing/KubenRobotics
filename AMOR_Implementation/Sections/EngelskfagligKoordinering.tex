\section{Engelskfaglig Koordinering} \label{Sec: Engelskfaglig_Koordinering}

AMOR vil basere seg på forkunnskaper gitt i engelskfagene på 1. og 2. trinn.
Disse forkunnskapene er introduserende elementer som gir et overblikk over robotikkfaget, og forbereder elevene på temaene som dekkes i AMOR.
Kuben vil primært forholde seg til «Spong et al. - Robot Modelling and Control» og «Siciliano et al. - Robotics Modelling, Planning and Control» i gjennomførelsen av AMOR, og disse bøkene vil kunne fungere som pensummateriale for engelskfaget.\\
De to delmålene for engelskfaglig koordinering i VG1 oppnås fullstendig ved å gjennomgå kapittel 1 i Spong. Til det første delmålet er det produsert supplement/støtte-materiale som finnes på ressurssiden, og kan benyttes som pensummateriale. Her er matematikken erstattet med ord og konseptene skrevet mer pedagogisk. For Vg2 dekkes de to delmålene nesten fullstendig ved kapittel 1 i Siciliano. Faglærer i robotikk kan supplere med manglende stoff.

Som et pedagogisk virkemiddel, og et motivasjonstiltak for studentene foreslås det å ha en muntlig høring med elevene, etter stoffet i engelskfaget er dekket.

	\subsection{Muntlig høring}
			Her ser man for seg en 1 til 1 dialog mellom en elev og en faglært, hvor eleven blir hørt i stoffet, med engelsklærer tilstede.
			Det er essensielt og avgjørende at det er de to delmålene med underpunkter som er fokuset for denne dialogen.
			I en slik dialog vil eleven ha mulighet til å stille spørsmål til teksten, og samtalen burde bære preg av kontinuerlig dialog.
			Dette vil gi muntlige vurderingsmuligheter for engelsklærer. Faglærer forbereder kontrollspørsmål for dialogen i samsvar med engelsklærer.
