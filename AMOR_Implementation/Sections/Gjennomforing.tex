\section*{Gjennomføring} \label{Sec: Gjennom}

    \begin{itemize}

        \item Før året starter burde det ses på og vurderes:

            \begin{itemize}
                \item Bruk av eksterne foredragsholdere som et motivasjonstiltak, samt et tiltak for bidrag til et kunnskapsnettverk for elevene.
                \item Ekskursjoner til relevante robotikkmiljøer og eller fabrikker som bruker roboter.
                \item Læringsressurser og strategi for faget, i henhold til vurderinger og feedback fra elevene, samt erfaringer gjordt av fjordårets faglærere.
                \item Bruk av Student Response Systems (SRS) som Mentimeter og Kahoot. En eventuell bruk burde forberedes godt og samspille med fagets læreplanmål.
            \end{itemize}

        \item I løpet av året bør elever oppfordres til å dele læringsressurser med resten av kullet og faglærere. Dette anbefales gjort på fagets GitHub side, i mappen ressurser.
        \item Slack benyttes som kommunikasjonsplattform i faget. Spesielt brukes Bitraf sitt workspace. Dette for å knytte elevene til makermiljøet, og gi dem kjenskap til jobbrelevante kommunikasjonsplattformer.
        \item En helhetlig introduksjon til robotikkfaget generelt og AMOR faget sett i denne sammenhengen, gis ved kursstart, dersom det ikke er gitt ila VG1 eller VG2
        \item Ved årsstart skal elevene reflektere på hvor de selv og robotikken er på vei hen. Dette skriftelig og indiviuelt. Spørsmålene gis på engelsk som en mulighet for tverrfaglighet, valgfritt svarspråk. Elevene skal få et ark hvor de svarer på følgende:

            \emph{Questions regarding life in general:}
            \begin{itemize}


                \item Where do you wanna go?
                \item Why do you wanna get there?
                \item How are you gonna get there?
                \item How are you gonna mark your progress?

            \end{itemize}
            In other words, what is your dream for your future?\\

            \emph{Questions regaring the course:}
            \begin{itemize}

                \item Do you have any expectations for this course? If so, describe them.
                \item On a scale of 1 to 100, how much robotics do you know?
                \item How would you rate your education in robotics so far?
                \item How do you learn best?

            \end{itemize}

            \emph{Questions regaring robotics:}
            \begin{itemize}

                \item Describe a desirable future with robots in our everyday life
                \item Describe a undesirable future with robots in our everyday life
                \item In what direction do you think robots are headed in the future?
                \item If a robot became conscious, how would you relate to that, and what do you believe it's rights should be?

            \end{itemize}
        Etter endt refleksjon er en hjemmelekse å se filmen Chappie, samt black mirror episoden Metalhead. Det tipses om seriene Humans og Westworld. Åpen diskusjon i klassen om temaene tatt opp i filmene tas i en senere klasseromstime.
        % https://en.wikipedia.org/wiki/Robot_ethics

        Etter diskusjon i klasseromstimen er et individuelt prosjekt å beskrive en nyhetsak innen fagområdet. Dette fremmer en undersøkende og kartleggende aproach, og prosjektet kan være en ny teknologisk løsning, et gjennombrudd eller et annen dagsrelevant tema innenfor robotikk. Temaet er fritt og ment som et motivasjonstiltak og et tiltak for å hjelpe elevene til å være oppdatert på området. Prosjektet fremføres for resten av klassen.

    \end{itemize}
