\section{Elevenes forkunnskaper fra Data og Elektronikk} \label{Sec: KoordDataEl}

    Forkunnskapene som forventes fra Data og Elektronikk dreier seg primært om elevenes evne til å programmere i språket C++.
    Her fins det mye gode ressurser på nett, forøvrig er det i \href{https://github.com/KvalheimRacing/KubenRobotics/tree/master/Resources}{ressursmappen på GitHub} lagt en lærebok i C++, som blant annet brukes på NTNU, og anbefales som faglitteratur. I tillegg til konseptene som nevnes, er det fordelaktig å komme inn på temaet parallellitet, når man snakker om C++.\\
    Når det gjelder bruk av biblioteker utenfor standarbiblotekene henvises det til \href{https://en.cppreference.com/w/cpp/links/libs}{denne} listen.\\
    Bruk av bibloteker utenfor denne listen er ikke anbefalt.\\
    Programeringsspråkene C, Python samt HTML og Javascript (grunnleggende webprogrammering) er fordelaktige å ha kjenskap til. C fordi språket sammenfaller med C++, Python fordi ROS noder både kan skrives i C++ og Python, og HTML samt Javascript for å ha muligheten til å modifisere enkle brukergrensesnitt for roboter.
